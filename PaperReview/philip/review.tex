\documentclass[a4paper]{article}
\usepackage[utf8]{inputenc}
\usepackage[english]{babel}
\usepackage[T1]{fontenc}
\usepackage{verbatim}


%\usepackage[left=2cm,right=2cm,top=2cm,bottom=2cm]{geometry}




\begin{document}
%\section*{}

Philip Holgersson, II-B, ada09pho@student.lu.se\\
Sarama, Redmiles, van der Hoek: ``Empirical Evidence of the Benefits of Workspace Awareness in Software Configuration Management'', 2008 \\
\textit{Relevance: X, Ease of review: X}

% =============================================================================
\section*{Summary}
% =============================================================================
%The paper I have read is about a study about the benefits of the use of workspace awareness tool in cooperation with a SCM tool for a project. The authors of the paper has as a goal to further prove that it is beneficial to use a workspace awareness tool when working in parallel with a SCM tool. To support their thesis they are adding more, and more detailed, evidence to strengthen already published work and also to get some results . They gather their evidence by conducting an experiment where they are using a workspace awareness tool \begin{comment}that is similar to others ones that are already reviewed\end{comment} and using this in a controlled environment while letting two groups of test subjects perform parallel work with a SCM tool. They perform two types of test with two types of conflicts that might emerge when working in parallel with a SCM tool.
%In the first test they use a text base kind of project to set a baseline in how the two different ways of working differs from each other. The seconds test is done with programming. The test groups are given different programming tasks to perform while the researchers simulates a future conflict. The idea of the tests is to see how many and in what way the two test groups handle conflicts that may appare during this kind of parallel work. 
The paper I have read is about giving proof that one can benefit from using a workspace awareness tool when working in parallel with software configuration management (SCM) tools, especially when dealing with conflicts that may appear during the work. There are studies that show that a lot of people working in this kind of environment often tell each other when they are performing a commit, because then at least their colleagues have been warned. There are some really good SCM tools developed, but the need for communication and potential conflicts are still there. That is what a workspace awareness tool is helping with and that is what the authors of this paper wants to prove.

The paper is to be a continuance and an elaboration of other papers that have addressed the same issue, with the big difference being that the workspace awareness tool used in this paper, Palantir,  is more powerful and more detailed according to previous studies and experiments.
The proof of the benefits of a workspace awareness tool in given through an experiment and the evaluation of that experiment. They are concentrating on how to detect and react to potential conflicts and problems that may emerge when working in parallel on the same artifacts. There are two types of conflicts they are looking at, direct conflicts and indirect conflicts. \begin{comment}Direct conflicts is when two people is working on the same artifact on different workspaces and their changes to the file needs to be merged to create a combined result. Indirect conflict is for example when person A is changing a file that person B uses as an import in another file. When person A commits and person B updates, person B's file does not work anymore because of the changes made in the imported file that person A was working on.\end{comment}
The experiment used in this paper in done with two test teams, a control team and an experiment team. The control team is using a SCM tool only and the experiment team is using the same SCM tool and Palantir. The experiment is divided into two parts a text part and a programming part. The text part of the experiment was a set of text-based assignments performed on a carefully chosen text. The test subject then performed a series of changes to the text with the intention that the test subjects simulates software changes. Also the basic behavior and features of Palantir was evaluated. The second part of the experiment used the workspace awareness tool in a programming domain, where the test subject would work in parallel with a shared code. Here a problem emerge that could potentially make the results indecisive. The authors wanted to see and to draw conclusions about how individuals behave in different situations. But since people have different skills in programming the conflicts that might emerge would not be the same for everyone partitioning in the experiment. To solve the problem and to be able to look at how individuals, but still in a team setting, the authors used so called confederates. That is research personnel that is acting the roll as a team member. This allowed the them not only to generate the same conflicts for all the participants, it also made it possible to generate the conflicts at the same intervals for all the participants. 
% =============================================================================
\section*{Evaluation}
% =============================================================================



% =============================================================================
\section*{Synthesis}
% =============================================================================



\end{document}