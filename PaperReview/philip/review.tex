\documentclass[a4paper]{article}
\usepackage[utf8]{inputenc}
\usepackage[english]{babel}
\usepackage[T1]{fontenc}
\usepackage{verbatim}


\usepackage[left=3cm,right=3cm,top=3cm,bottom=3cm]{geometry}




\begin{document}
\section*{Paper review}

Philip Holgersson, II-B, ada09pho@student.lu.se\\
Sarama, Redmiles, van der Hoek: ``Empirical Evidence of the Benefits of Workspace Awareness in Software Configuration Management'', 2008 \\
\textit{Relevance: 3, Ease of review: 4}

% =============================================================================
\subsection*{Summary}
% =============================================================================

The paper I have read is about giving proof that one can benefit from using a workspace awareness tool when working in parallel with software configuration management (SCM) tools, especially when dealing with conflicts that may appear during the work. There are studies that show that a lot of people working in this kind of environment often tell each other when they are performing a commit, because then at least their colleagues have been warned. There are some really good SCM tools developed, but the need for communication and potential conflicts are still there. That is what a workspace awareness tool is helping with and that is what the authors of this paper wants to prove.

The paper is to be a continuance and an elaboration of other papers that have addressed the same issue, with the big difference being that the workspace awareness tool used in this paper, Palantir,  is more powerful and more detailed according to previous studies and experiments.
The proof of the benefits of a workspace awareness tool in given through an experiment and the evaluation of that experiment. They researchers are concentrating most on how using a workspace awareness tool can improve someone's abilities to detect and react to potential conflicts and problems that may emerge when working in parallel on the same artefacts. They also wants to see how using the tool affects time-to-completion of tasks and if it is promoting more coordination between team members.

There are two types of conflicts they are looking at, direct conflicts and indirect conflicts. 
The experiment used in this paper in done with two test teams, a control team and an experiment team. The control team is using a SCM tool only and the experiment team is using the same SCM tool and Palantir. The experiment is divided into two parts a text part and a programming part. The text part of the experiment was a set of text-based assignments performed on a carefully chosen text. The test subject then performed a series of changes to the text with the intention that the test subjects simulates software changes. Also the basic behaviour and features of Palantir was evaluated. The second part of the experiment used the workspace awareness tool in a programming domain, where the test subject would work in parallel with a shared code. Here a problem emerge that could potentially make the results indecisive. The authors wanted to see and to draw conclusions about how individuals behave in different situations. But since people have different skills in programming the conflicts that might emerge would not be the same for everyone partitioning in the experiment. To solve the problem and to be able to look at how individuals, but still in a team setting, the authors used so called confederates. That is research personnel that is acting the roll as a team member. This allowed the them not only to generate the same conflicts for all the participants, it also made it possible to generate the conflicts at the same intervals for all the participants.

The result from the experiment was rather convincing in that the workspace awareness tool is very beneficial when working in parallel in combination with a SCM tool. The results show that it makes a big difference when it comes to detecting (and solving) both direct and indirect conflicts before they actually appear, while the time-to-completion is a bit longer than without. However, this extra time provides another benefit since the quality of the text in the first part of the experiment was significantly higher due to the fact that it contained a lot less unattended conflicts when delivered. For the second part the researchers found that the extra time was again found to be legit since the final code did not contain any indirect conflicts at all. Lastly they also found a correlation between the number of resolved conflicts and the number of coordination actions taken. The final verdict is then that using a workspace awareness tool in combination with a SCM tool can have a huge beneficial impact on a team. One do have one more software to learn to use and to keep an eye on when working and it might take a little longer for each task, but the quality of the work will probably be higher and less time is needed to fix conflicts in retrospect.
% =============================================================================
\subsection*{Evaluation}
% =============================================================================
A workspace awareness tool is a help to communicate with ones team members, something that often forgotten. Communication is a key when working in a team and on the same artefacts. This paper and the experiment it describes is done to get more people to communicate better. When working in a small team where the whole team is near each other, some would say that a workspace awareness tool might be unnecessary since they can easily talk to each other about who is doing what and so on. However, just talking to each other does not give an overview of the potential conflicts that this kind of tool does. The paper helps understand how and why a workspace awareness tool is beneficial and that if used it can help raise quality and reduce time spend fixing things in retrospect.
I think that this paper gives a case for using a workspace awareness tool and according to earlier work in the same field the tool the authors used, Palantir, seams to be very good. Of course there are some drawbacks, the fact that one have yet another software to learn and to keep track on while working is not forgotten.

In the result section of the paper is where the theory is realized. They show a huge difference between using and not using a tool, almost too good to be true. To make the findings even more firm the authors probably should have conducted the experiment with a higher number of people or performed the experiment multiple times. In this case it is one experiment (with two parts) using Information and Computer Science students at an University. It would have been interesting, and more convincing if they have conducted the exact same experiment on a real company that deals with these kind of problems on a daily basis and have more experience in dealing with conflicts. Non the less, the paper raises the idea of more use of workspace awareness tools and why it is a good idea to use them and I think the authors did that in an overall good way.


% =============================================================================
\subsection*{Synthesis}
% =============================================================================

One problem that this paper highlights is the difference of individual skill if people one has to tend to when evaluating a software tool. The difference in skill has the ability to effect the tool and it is then hard to get a fair judgement of the tool. The use of confederates reduces this a bit but the problem still exists and I can not think of a way to fully eliminate it. Another crux to is worth mentioning is that the tool that is evaluated in this paper requires a visual developing environment. If someone is exclusively using a terminal working area these tools are not very useful.

To argue that the paper and the experiment is not complete and does not make a sufficiently strong case one can say that this experiment is done by a small team with a minor project and it is not performed in real life environment. That might be true and as I mentioned earlier, to get a better evaluation of the workspace awareness tools, a more elaborate experiment would have been better and provided more real life results. This is also something the authors is aware about and discuss.

Throughout the course so far, in every discussion and exercise I have had, the topic of communication has been involved. I think that shows that the existence and use of a workspace awareness tool is connected to the course. Of course the course is about much more than that and communication and communication tools is just a minor part of the course but it is still there.

Our group has not yet decided what project we are doing for this course, so there is no specific topic to relate this paper to. However, the need to communicate good is necessary in every project and since we are writing a report on the project that we will work in parallel on, and that is very much what this paper is about. Communication when working in parallel in a team and on the same artefacts.

\end{document}